% http://www.ctan.org/tex-archive/macros/latex/contrib/beamer/examples
% http://latex.artikel-namsu.de/english/beamer-examples.html

%\documentclass{beamer}
\documentclass[usenames,dvipsnames]{beamer}
\usepackage{amsmath}
\usepackage{amssymb}
\usepackage{bm}
\usepackage{fancybox, graphicx}
\usepackage{listings}
\usepackage{tikz} % Diagrams
\usepackage{color}
\usepackage{textcomp} % See https://tex.stackexchange.com/questions/145416/how-to-have-straight-single-quotes-in-lstlistings

\lstset{language=bash,upquote=true} % Format listings as appropriate for bash. Inexplicably we get problems if the language is set as part of the \begin{lstlisting} command.

% https://tex.stackexchange.com/questions/36030/how-to-make-a-single-word-look-as-some-code
\definecolor{light-gray}{gray}{0.95}
\newcommand{\code}[1]{\colorbox{light-gray}{\texttt{#1}}}
\newcommand{\insertpicture}[3]{{\setbeamercolor{background canvas}{bg={#3}}\begin{frame}{} \begin{center}\includegraphics[{#2}]{{#1}} \end{center} \end{frame}}}



\usetheme{boxes}
\usecolortheme{beaver}


\title{Can the World's Largest Digital Camera Answer Cosmological Questions?}
\author{Lorne Whiteway \\ lorne.whiteway@star.ucl.ac.uk}
\institute{Astrophysics Group \\ Department of Physics and Astronomy \\ University College London}
\date{Presentation to the Orwell Astronomical Society \\ 21 September 2018 \\ Find the presentation at \alert{\url{https://tinyurl.com/y7w542eb}}}

\subject{IT}

\begin{document}

\frame{\titlepage}

\insertpicture{11-0222-13D.jpg}{angle=-90,origin=c,scale=0.3}{black}

\insertpicture{11-0200-15D.jpg}{scale=0.40}{black}


\begin{frame}{Detector}
  \begin{block}{}
    \begin{itemize}
      \item{Detector has 62 chips (`CCDs')}
      \item{Each CCD is 3 cm by 6 cm and has $2048 x 4096 = 8 \text{ megapixels}$}
      \item{Total of $500$ megapixels.}
      \item{Each pixel is $15$ microns square.}
      \item{The CCDs are unusually thick $\implies$ more infrared light captured.}
      \item{\color{blue}{Why do we want to capture infrared light?}}
    \end{itemize}
  \end{block}
\end{frame}

\insertpicture{DECam_in_place.jpg}{scale=1.4}{black}

\insertpicture{DECam-overview.png}{scale=0.75}{black}

\begin{frame}{DECam}
  \begin{block}{}
    \begin{itemize}
      \item{Detector is part of a camera called `DECam'.}
      \item{Built in part at University College London.}
      \item{Five lenses - largest is $1$m diameter.}
      \item{Careful shutter design allows precise measurement of exposure times.}
      \item{Five filters: green, red, and three infrared colours.}
      %\item{\color{blue}{Why do we want to see in several different colors?}}
    \end{itemize}
  \end{block}
\end{frame}

% Plot of total throughput for each band (includes filter, apparatus and atmosphere)
\insertpicture{dr1_bandpass.jpg}{scale=0.27}{white}

% Telescope from outside
\insertpicture{12-0328-26D.jpg}{scale=0.35}{black}

% Telescope from inside
\insertpicture{12-0331-07D.jpg}{scale=0.20}{black}

\begin{frame}{The Telescope}
  \begin{block}{}
    \begin{itemize}
      \item{Camera is attached to the Victor Blanco Telescope}
      \item{At the Cerro-Tololo Inter-American Observatory in Chile}
      \item{4m main mirror; $10 m^2$ collecting area}
      \item{First light 1976; largest Southern Hemisphere telescope until 1998}
      \item{At 2200 m altitude}
      \item{Ritchey-Chr\'etien design}
    \end{itemize}
  \end{block}
\end{frame}

\begin{frame}{Optical system: Telescope plus camera}
  \begin{block}{}
    \begin{itemize}
      \item{Camera is at prime focus.}
      \item{f2.7}
      \item{Field of view: 2 deg diameter; $3 deg^2$ area.}
      \item{Pixel scale is $0.26$ arcsec/pixel (pound coin at 18 km). }
    \end{itemize}
  \end{block}
\end{frame}

\end{document}
