% http://www.ctan.org/tex-archive/macros/latex/contrib/beamer/examples
% http://latex.artikel-namsu.de/english/beamer-examples.html

%\documentclass{beamer}
\documentclass[usenames,dvipsnames]{beamer}
\usepackage{amsmath}
\usepackage{amssymb}
\usepackage{bm}
\usepackage{fancybox, graphicx}
\usepackage{listings}
\usepackage{tikz} % Diagrams
\usepackage{color}
\usepackage{textcomp} % See https://tex.stackexchange.com/questions/145416/how-to-have-straight-single-quotes-in-lstlistings

\lstset{language=bash,upquote=true} % Format listings as appropriate for bash. Inexplicably we get problems if the language is set as part of the \begin{lstlisting} command.

% https://tex.stackexchange.com/questions/36030/how-to-make-a-single-word-look-as-some-code
\definecolor{light-gray}{gray}{0.95}
\newcommand{\code}[1]{\colorbox{light-gray}{\texttt{#1}}}
\newcommand{\insertpicture}[3]{{\setbeamercolor{background canvas}{bg={#3}}\begin{frame}{} \begin{center}\includegraphics[{#2}]{{#1}} \end{center} \end{frame}}}



\usetheme{boxes}
\usecolortheme{beaver}


\title{Can the World's Largest Digital Camera Answer Cosmological Questions?}
\author{Lorne Whiteway \\ lorne.whiteway@star.ucl.ac.uk}
\institute{Astrophysics Group \\ Department of Physics and Astronomy \\ University College London}
\date{Presentation to the Orwell Astronomical Society \\ 21 September 2018 \\ Find the presentation at \alert{\url{https://tinyurl.com/y7w542eb}}}

\begin{document}

\frame{\titlepage}

% Blank screen, ask who has digital camera, show full scale printout of focal plane. Explain how the detector works, and how many pixels it has.

% From https://www.darkenergysurvey.org/multimedia/photo-gallery/
\insertpicture{11-0222-13D.jpg}{angle=-90,origin=c,scale=0.3}{black}

% From https://www.darkenergysurvey.org/multimedia/photo-gallery/
\insertpicture{11-0200-15D.jpg}{scale=0.40}{black}


% Don't show this slide - talk through this material during the preceeding pictures
%\begin{frame}{Detector}
%  \begin{block}{}
%    \begin{itemize}
%      \item{Detector has 62 chips (`CCDs')}
%      \item{Each CCD is 3 cm by 6 cm and has $2048 x 4096 = 8 \text{ megapixels}$}
%      \item{Total of $520$ megapixels.} % Explain that each pixel counts the number of photons impinging on the pixel.
%      \item{Each pixel is $15$ microns square.}
%      \item{The CCDs are unusually thick $\implies$ more infrared light captured.}
%      \item{\color{blue}{Why do we want to capture infrared light?}}
%    \end{itemize}
%  \end{block}
%\end{frame}


%http://vms.fnal.gov/stillphotos/2012/0300/12-0332-18D.jpg
\insertpicture{12-0332-18D.jpg}{scale=0.4}{black}

%Schematic diagram of DECam
% From paper `DECam Integration Tests on Telescope Simulator' http://inspirehep.net/record/946923. See also https://arxiv.org/pdf/1111.4717.pdf.
\insertpicture{DECam-overview.png}{scale=0.75}{black}

% Don't show this slide - talk through this material during the preceeding pictures
%\begin{frame}{DECam}
%  \begin{block}{}
%    \begin{itemize}
%      \item{Detector is part of a camera called `DECam'.}
%      \item{Built in part at University College London.}
%      \item{Five lenses - largest is $1$m diameter.}
%      \item{Five filters: green, red, and three infrared colours.}
%    \end{itemize}
%  \end{block}
%\end{frame}

% Plot of total throughput for each band (includes filter, apparatus and atmosphere)
% From http://www.ctio.noao.edu/noao/node/13140
\insertpicture{dr1_bandpass.jpg}{scale=0.27}{white}

% Telescope from outside
% From DES Fermilab Gallery at http://vms.fnal.gov/asset?search-term=%22dark+energy+survey%22&submit=Go&search-category%5B%5D=Photography
\insertpicture{12-0328-26D.jpg}{scale=0.35}{black}

% Telescope from inside
% From DES Fermilab Gallery at http://vms.fnal.gov/stillphotos/2012/0300/12-0331-04D.jpg
\insertpicture{12-0331-04D.jpg}{scale=0.4}{black}

\begin{frame}{The Telescope}
  \begin{block}{}
    \begin{itemize}
      \item{The DECam camera is attached to the Victor Blanco Telescope at the Cerro-Tololo Inter-American Observatory in Chile}
      \item{4 m main mirror; 10 m$^2$ collecting area}
      \item{First light 1976; largest Southern Hemisphere telescope until 1998}
      \item{At 2200 m altitude}
      \item{Ritchey-Chr\'etien design}
    \end{itemize}
  \end{block}
\end{frame}

\begin{frame}{Optical system: Telescope plus camera}
  \begin{block}{}
    \begin{itemize}
      \item{The camera is at prime focus.}
      \item{f2.7}
      \item{Field of view: 2 deg diameter; 3 deg$^2$ area.}
      \item{$0.26$ arcsec per pixel (`pixel scale'). } %$0.26$ arcsec = pound coin at 18 km
    \end{itemize}
  \end{block}
\end{frame}

% Show off smaller scale picture. Use DS9 with `Linear' and `ZScale' under scale; InvertY and 270 under Zoom; InvertColorMap under Color.
% Scale, Scale Parameters, set min to 1.5 times proposed value and max to 5 times proposed value. Then expiriment with right mouse button.
% Zoom: InvertY and 270 degrees.

% Talk about star/galaxy separation and machine learning.
% Talk about bleeding from bright objects.

% Perhaps show off big scale picture?

%From https://files.slack.com/files-pri/T04GF69PU-FCDLYPTHP/des0303-3540-spiral.jpg, via DES Slack beautiful-pictures post 23 August 2018.
\insertpicture{des0303-3540-spiral.jpg}{scale=0.28}{black}


\begin{frame}{Dark Energy Survey}
  \begin{block}{}
    \begin{itemize}
      \item{The camera is being used as part of a survey to collect information about the locations of many distant galaxies.}
      \item{Expect to see 300 million galaxies.}
      \item{We only do statistical analysis on the data - we don't actually care about the details of any one particular object.}
    \end{itemize}
  \end{block}
\end{frame}

\begin{frame}{Dark Energy Survey}
  \begin{block}{}
    \begin{itemize}
      \item{Survey lasts six years - year six is just starting.}
      \item{Survey covers one-eighth of the celestial sphere.}
      \item{Each patch imaged ten times with each of the five filters.}
      \item{Each exposure is 90 seconds.} %500 hours per year.
    \end{itemize}
  \end{block}
\end{frame}

%TODO: Picture of survey area? names of constellations?

%Perhaps don't include this slide...
%\begin{frame}{Dark Energy Survey}
%  \begin{block}{}
%    \begin{itemize}
%      \item{The project involves over 400 scientists at 25 institutions in 7 countries.}
%      \item{Funding is primarily from the U.S. Department of Energy.}
%      \item{U.K. institutions are UCL and the Universities of Cambridge, Edinburgh, Portsmouth, Sussex and Nottingham.}
%    \end{itemize}
%  \end{block}
%\end{frame}

\begin{frame}{Cosmological redshift}
  \begin{block}{}
    \begin{itemize}
      \item{It's easy to measure the sky coordinates RA and DEC of each object.}
      \item{But we also want to know how far away the object is, to determine its place in three-dimensional space.}
      \item{The expansion of the Universe `stretches' lightwaves, making the wavelength longer (redder). This is `cosmological redshift'.}
      \item{From the redshift we can infer the distance (more redshift $\implies$ more distant).}
    \end{itemize}
  \end{block}
\end{frame}

% Can I find a picture of cosmological redshift to add here?

\begin{frame}{Photometric redshifts}
  \begin{block}{}
    \begin{itemize}
      \item{If we could point a spectrograph at each object, then we could precisely measure the redshift noting how much the spectral lines have shifted.}
      \item{This would take too long!}
      \item{But we get some (very coarse) spectral (i.e. colour) information by measuring the brightness through each of the five filters.}
      \item{From this we can get a `good enough' estimate of the redshift.}
      \item{What can go wrong: small old nearby red galaxy and large old distant blue galaxy are indistinguishable.}
    \end{itemize}
  \end{block}
\end{frame}

\begin{frame}{Why a survey?}
  \begin{block}{}
    \begin{itemize}
      \item{So what do we do with all these galaxy positions?}
    \end{itemize}
  \end{block}
\end{frame}

\begin{frame}{Cosmology}
  \begin{block}{}
    \begin{itemize}
      \item{Cosmology is the study of the Universe on its largest scales.}
    \end{itemize}
  \end{block}
\end{frame}

\begin{frame}{Cosmological questions}
  \begin{block}{}
    \begin{enumerate}
      \item{Did the Universe have a beginning and if so old is it now?}
      \item{Is the Universe expanding and if so how fast?}
      \item{What types of matter and energy predominate in the Universe and what are their densities?}
      \item{What is the mass of the neutrino?}
    \end{enumerate}
  \end{block}
\end{frame}

\begin{frame}{Cosmological questions that we don't work on}
  \begin{block}{}
    \begin{enumerate}
      \item{How big is the Universe?}
      \item{What caused the Big Bang?}
      \item{Are there other Universes?}
    \end{enumerate}
    We (currently) have no tools to use to answer these questions.
  \end{block}
\end{frame}


\begin{frame}{First principles}
  There is strong evidence that:
  \begin{block}{}
    \begin{enumerate}
      \item{There was a Big Bang - an initial uniformly hot and dense state - and the Universe has been expanding ever since.}
      \item{The Universe is more-or-less the same everywhere and we are not in a `special' location.}
      \item{Einstein's theory (`General Relativity') correctly describes how gravity works.}
      \item{The overall geometry of the Universe is `flat': keep going in a straight line and you won't return home.}
    \end{enumerate}
  \end{block}
\end{frame}


\begin{frame}{So how can we answer these cosmological questions?}
  \begin{block}{}
    \begin{itemize}
      \item{One main method is to look at how `clustered' galaxies are.}
      \item{Galaxies aren't randomly distributed through space - instead, they cluster together under the influence of gravity.}
    \end{itemize}
  \end{block}
\end{frame}

%Pandora's Cluster. 4 billion light years.
% http://hubblesite.org/image/3868/category/52-frontier-fields
% Abell 2744
\insertpicture{_STScI-gallery-1401a-480x630.jpg}{scale=0.4}{black}

% If you want another picture, try hubble3-hubblefirstfrontier_1.jpg


\begin{frame}{Clustering}
  \begin{block}{}
    \begin{itemize}
      \item{The older the Universe, the longer time gravity has had to operate, so the more clustering.}
      \item{The more stuff in the Universe, the more gravity, so the more clustering.}
      \item{The two effects are similar but distinguishable.}
    \end{itemize}
  \end{block}
\end{frame}


\begin{frame}{So here's the plan}
  \begin{block}{}
    \begin{enumerate}
      \item{Agree on a definition of clustering.}
      \item{Theoretical astrophysicists calculate how much clustering they would expect for a range of ages and densities.}
      \item{Astronomers measure how much clustering there actually is.}
      \item{We match the results to see which age/density combination makes theory equal observation.}
    \end{enumerate}
  \end{block}
\end{frame}



%https://www.kingsnews.org/articles/the-cocktail-party-effect
\insertpicture{party.jpg}{scale=0.55}{black}


\begin{frame}{Definition of clustering}
  \begin{block}{}
    \begin{itemize}
      \item{Measure the distance between each pair of objects. Expect to see lots of pairs with small separation.}
      \item{Draw a histogram of the results.}
      \item{Repeat using randomly positioned objects.}
      \item{Look at the percentage difference between the two histograms.}
    \end{itemize}
  \end{block}
\end{frame}


\begin{frame}{Calculating how much clustering we would expect in theory}
  \begin{block}{}
    \begin{itemize}
      \item{Details of this are beyond the scope of this talk.}
      \item{See Dodelson's book \textit{Modern Cosmology} for details.}
      \item{Ingredients: theory of gravity, interactions between light and electrons and between electrons and protons (important in early Universe).}
      \item{Mathematical tools: Perturbation theory, Fourier analysis.}
      \item{Can check the results using computer simulations.}
    \end{itemize}
  \end{block}
\end{frame}


% Do Clustering Exercise
\begin{frame}{Exercise: You are the Cosmologist}
  \begin{block}{}
    \begin{itemize}
      \item{Some of you have been given some real galaxy positions and some of you have been given some random positions.}
      \item{Calculate how much clustering.}
      \item{Compare your results to the theoretical results.}
      \item{How old is the Universe? What is the density of matter?}
    \end{itemize}
  \end{block}
\end{frame}

% My own work...
\insertpicture{ClusteringTables.jpg}{scale=0.55}{white}



\begin{frame}{Conclusions of the exercise}
  \begin{block}{}
    \begin{itemize}
      \item{Universe is about 14 billion years old.}
      \item{Density of matter is about 4 g per Jupiter volume.}
    \end{itemize}
  \end{block}
\end{frame}

\begin{frame}{Comments}
  \begin{block}{}
    \begin{itemize}
      \item{\color{blue}{Why was it important that all the galaxies have the same redshift?}}
    \end{itemize}
  \end{block}
\end{frame}


% My own work...
\insertpicture{ExamplePosterior.png}{scale=0.65}{white}


\begin{frame}{What types of stuff?}
  \begin{block}{}
    \begin{itemize}
      \item{Our actual analysis is more complicated.}
      \item{The theoreticians actually create clustering graphs for a huge range of densities for different possible types of `stuff'. }
    \end{itemize}
  \end{block}
\end{frame}

\begin{frame}{Dark matter}
  \begin{block}{}
    \begin{itemize}
      \item{They include the possibility that some of the matter doesn't interact with light.}
	\item{Such `dark matter' doesn't cluster as easily as ordinary matter. (Why not?)}
      \item{This gives a recognisable signature in the clustering graph.}
    \end{itemize}
  \end{block}
\end{frame}

\begin{frame}{Dark energy}
  \begin{block}{}
    \begin{itemize}
      \item{They also include the possibility that empty space itself has some mass.}
	\item{This mass is called `dark energy'.}
      \item{This mass is everywhere, and can't cause clustering.}
    \end{itemize}
  \end{block}
\end{frame}
      
\begin{frame}{Conclusions}
  \begin{block}{}
    \begin{itemize}
      \item{If we throw all these stange things into the range of possibilities, then we find that the best match to the observed clustering is:}
	\item{Age of Universe = 14 billion years (as before).}
      \item{Of the 4.3 g per Jupiter volume of matter, only 0.7 g is normal matter (basically hydrogen and helium) and 3.6 g is dark.}
      \item{As well, there is an additional 9 g per Jupiter volume that is simply the mass of empty space.}
    \end{itemize}
  \end{block}
\end{frame}

\begin{frame}{Dark Energy}
  \begin{block}{}
    \begin{itemize}
      \item{Dark energy is not well understood.}
	\item{Normally you must expend energy to increase a volume (think of the piston on a steam locomotive).}
      \item{But dark energy \textit{increases} as space expands. Thus we say that dark energy has a \textit{negative} pressure.}
    \end{itemize}
  \end{block}
\end{frame}

\begin{frame}{Cosmic acceleration}
  \begin{block}{}
    \begin{itemize}
      \item{Analysis of the Dark Energy Survey results shows that this pressure is exactly the negative of the energy density (the two quantities have the same units).}
      \item{This negative pressure causes an \textit{acceleration} in the expansion of space.}
      \item{This acceleration was first observed in the 1990s.}
      \item{The accelaeration has been slowly building for the last five billion years.}
      \item{This will dominate the future Universe - in the distant future our galaxy will have no near neighbours.}
    \end{itemize}
  \end{block}
\end{frame}

\begin{frame}{Further reading}
  \begin{block}{}
    \begin{itemize}
      \item{Harrison, \textit{Cosmology}, Cambridge University Press, 2nd ed 2000. Historical cosmology and the philosophy of cosmology as well as modern theory. Non-mathematical.}
      \item{Hawley \& Holcomb, \textit{Foundations of Modern Cosmology}, OUP, 2005. Undergraduate textbook.}
      \item{Dodelson, \textit{Modern Cosmology}, Academic Press, 2003. Introduction to the mathematical theory; requires undergraduate physics preparation.}
    \end{itemize}
  \end{block}
\end{frame}


\begin{frame}{Images are publicly available}
  \begin{block}{}
    \begin{itemize}
      \item{Go to \url{http://archive.noao.edu/search/query/survey/desy1}}
      \item{Set coordinates to (say) RA = 35, DEC = -50; Search box size = 20; choose a filter colour and check `Calibrated images'. Then `Search'.}
      \item{On the next page click on `Retrieve'.}
      \item{This will download a file ($\sim$ 300 Mb) that you can view with DS9.}
    \end{itemize}
  \end{block}
\end{frame}

\insertpicture{NOAO_1.jpg}{scale=0.35}{white}

\insertpicture{NOAO_2.jpg}{scale=0.35}{white}

\begin{frame}{Other uses of the data}
  \begin{block}{}
    \begin{itemize}
      \item{The images capture everything in the sky (stars, galaxies, solar system objects, cosmic rays, airplanes, etc.)}
      \item{Non-cosmologists are using the images e.g. to search for planet 9.}
    \end{itemize}
  \end{block}
\end{frame}

% TODO: Could add a slide about systematic errors.Systematic errors?
% Could add a slide about the CMB.

\end{document}
